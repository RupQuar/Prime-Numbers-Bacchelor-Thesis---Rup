\documentclass[german,12pt,a4paper]{article}
\usepackage[backend=biber,
style=alphabetic,
]{biblatex}
\addbibresource{literatur.bib}

\usepackage{soul}
\usepackage{amsmath}
\usepackage{mathtools}
\usepackage{amssymb}
\usepackage{amsthm}
%\usepackage{sagetex}

% set font encoding for PDFLaTeX, XeLaTeX, or LuaTeX
\usepackage{ifxetex,ifluatex}
\if\ifxetex T\else\ifluatex T\else F\fi\fi T%
  \usepackage{fontspec}
\else
  \usepackage[scaled=Wert]{helvet}
  \usepackage[T1]{fontenc}
  \usepackage[utf8]{inputenc}
  \usepackage{lmodern}
  \usepackage[ngerman]{babel}
\fi

\usepackage{hyperref}
\usepackage{parskip}
\usepackage{csquotes}

\title{Bachelorarbeit\\Prime Time for a Prime Number}
\author{Rupertus Weigner}

% Enable SageTeX to run SageMath code right inside this LaTeX file.
% http://doc.sagemath.org/html/en/tutorial/sagetex.html
% \usepackage{sagetex}

% Enable PythonTeX to run Python – https://ctan.org/pkg/pythontex
% \usepackage{pythontex}

\begin{document}

\maketitle
\thispagestyle{empty}
\newpage

\pagenumbering{roman}
\section*{Vorwort}
\addcontentsline{toc}{section}{\protect{}Vorwort}
Die Bachelor Arbeit verfasse ich im Themenbereich der Primzahlen. Ausgangspunkt die für diese These ist, dass am 7. Januar 2016 die damals größte Primzahl in einem Computer Labor, am Satelliten Campus an der University of Central Missouri, entdeckt worden ist. Die New York Times hat am 21. Januar 2016 diese großartige Entdeckung veröffentlicht und sie als "bigP"\ betitelt.\\
\begin{center}
$bigP_{2016} = 2^{74.207.281}-1$ $\rightarrow$ 2016 von Cooper, u.a.
\end{center}\
\\Diese Primzahl ist zu groß um ausgeschrieben zu werden. Um es sich bildlich besser vorstellen zu können würde diese Zahl 6.000 bis 7.000 DIN-A4-Blätter füllen können. Dies ist abhängig von der Schriftgröße. Um diese Zahl von menschenhand auszuschreiben würde eine Person mehr als drei Monate dafür benötigen. Eine solche Kombination an Zahlen nennt man "Mersenne-Primzahl"\ (The New York Times 2016).\
Im Laufe dieser Arbeit wurden jedoch 2 weitere größere Mersenne-Primzahlen entdeckt:\\
\begin{center}
$M_{77.232.917} = 2^{77.232.917}-1$ $\rightarrow$ 2017 von Jonathan Pace (GIMPS)\\
$M_{82.589.933} = 2^{82.589.933}-1$ $\rightarrow$ 2018 von Patrick Laroche (GIMPS)
\end{center}\
\\Diese Art von Primzahlen sind besonders, da in dieser Schriftweise $2^n-1$ erst 51 Primzahlen gefunden wurden (Austromath). ... Vielleicht ist das auch die Einleitung.
\newpage

\tableofcontents
\thispagestyle{empty}
\newpage

\section*{Erklärung der Symbole}
\addcontentsline{toc}{section}{\protect{}Erklärung der Symbole}
Hier werden die in dieser Arbeit verwendeten und beschriebenen Symbole für einen schnellen Überblick erklärt:\\
\begin{tabbing}
xxxxxxxxxxxxxxxxxxxxxxx\=xxxxxxxxxxxxxxxxxxxxxxx\kill
\large Symbol           \> \large Erläuterung\\
\\$\mathbb{N}$          \> natürliche Zahlen\\
\\$\mathbb{Z}$          \> ganze Zahlen\\
\\$\mathbb{Q}$          \> rationale Zalen\\
\\$\mathbb{I}$          \> irrationale Zahlen\\
\\$\mathbb{R}$          \> reelle Zahlen\\
\\ggT                   \> größter gemeinsamer Teiler\\
\\kgV                   \> kleinstes gemeinsames Vielfaches\\
\\n $\in$ $\mathbb{N}$  \> das Element n ist in $\mathbb{N}$ enthalten\\
\\$k \mid n$            \> die Zahl $k$ teilt die Zahl $n$\\
\\$k \nmid n$           \> die Zahl $k$ teilt die Zahl $n$ nicht\\
\\$\lceil{x}\rceil$     \> kleinste ganze Zahl größer oder gleich $x$\\
\\:=                    \> definitionsgemäß\\
\\$a\equiv b$ (mod $n$) \> $a$ ist kongruent zu $b$ modulo $n$

\end{tabbing}
\newpage

\section*{Einleitung}
\addcontentsline{toc}{section}{\protect{}Einleitung}
... wird zum Schluss verfasst!!!\newpage

\pagenumbering{arabic}
\setcounter{section}{0}
\setcounter{page}{1}
\section{Einführung in die elementare Zahlentheorie}
Bevor wir in die komplexe Materie der Primzahlen eintauchen können muss in erster Linie erörtert werden aus was sich Zahlen zusammensetzen und nach welchen Rechenregeln sie zerlegbar oder zu vervielfachen sind. Der nächste Abschnitt wird vielen Lesern und Leserinnen noch von der Schulzeit bekannt vorkommen. Die folgenden Seiten werden teils Auffrischung, teils neu sein und werden einen guten Einblick in die Zahlentheorie ermöglichen. Beschrieben wird in Kürze unter anderem in wie weit man Zahlen eingliedern kann und welchen Rechenregeln sie unterliegen.

\subsection{Mathematische Grundregeln}
%So Eigenschaften wie Assoziativitaet, Kommutativitaet, Distributivitaet waeren hier noch hilfreich
Aus dem Schulunterricht kennt man die klassischen arithmetischen Operationen, wie Addition, Subtraktion, Multiplikation und Division. Die Berechnung dieser sind simpler Natur und lauten:
\begin{itemize}
\item \textbf{Addition} $\to$ $x = x_{1} + x_{2} + \cdot\cdot\cdot + x_{n}$
\[\sum_{i=1}^n x_{i} = x_{1} + x_{2} +  \cdot\cdot\cdot + x_{n}\]

\item \textbf{Subtraktion} $\to$ $x = x_{1} - x_{2} - \cdot\cdot\cdot - x_{n}$

\item \textbf{Multiplikation} $\to$ $x = x_{1} \cdot x_{2} \cdot\cdot\cdot x_{n}$
\[\prod_{i=1}^n x_{i} = x_{1} \cdot x_{2} \cdot\cdot\cdot x_{n}\]

\item \textbf{Division} $\to$ $x = x_{1} : x_{2}$
\end{itemize}

Etwas komplizierter hingegen ist das \textbf{Quadrieren}. Quadrieren bedeutet, dass man eine Zahl oder Variable mit sich selbst multipliziert.
\begin{itemize}
\item Quadrieren von Zahlen: $3 \cdot 3 = 3^2 = 9$
\item Quadrieren von Variablen: $x \cdot x = x^2$
\end{itemize}
Die Thematik des Quadrierens wird im späteren Verlauf dieser Arbeit nochmals aufgegriffen und vertieft. 

\subsection{Eingliederung von Zahlen}
%Hier wäre ein Hinweis auf die formale Definition der nat. Z., etwa durch die Peano-Axiome, oder das John-von-Neumann-Modell möglich.
Zum Einen gibt es \textbf{natürliche Zahlen} - $\mathbb{N}$ - \{0, 1, 2, 3, 4, ...\}. Man erkennt, dass natürliche Zahlen bei der Zahl 0 beginnen und weiters ausschließlich aus nicht negativen ganzen Zahlen zusammengesetzt sind. Diese Zahlenreihe lässt sich beliebig lang weiterführen. Hier gibt es weitere Gliederungen:\
\begin{itemize}
\item $\mathbb{N}^*$ - sind alle natürlichen Zahlen ohne der Zahl 0
\item $\mathbb{N}_g$ - beinhaltet alle natürlichen geraden Zahlen
%erhaelt man durch Multiplikation aller Elemente aus N mit 2
\item $\mathbb{N}_u$ - beinhaltet alle natürlichen ungeraden Zahlen
%erhaelt man durch die Operation 2n-1 auf N*
\end{itemize}
Da die Zahlen von $\mathbb{N}$ aus rein positiven Zahlen 
%was nun? positive oder nicht negative Zahlen? 0 ist nicht positiv
%Du koenntest das auch explizit angeben, z.b. Die Subtraktion zweier Zahlen m, n aus N ist definiert, wenn m <= n. Die Division wenn m | n
bestehen ist das Rechnen mit diesen nur eingeschränkt möglich. Zum Einen erlauben sie es nicht willkürlich Zahlen zu subtrahieren und zum Anderen können diese nicht zufällig miteinander dividiert oder gar Wurzeln gezogen werden.

\textbf{Ganze Zahlen} - $\mathbb{Z} = \mathbb{N} \cup \mathbb\{-n | n \in \mathbb{N}\}$. Diese setzen sich wie die natürlichen Zahlen nur aus 
%??? das definierst du ja gerade.
ganzen Zahlen zusammen, jedoch bestehen diese im Gegensatz zu natürlichen Zahlen nicht nur aus % nicht negativen. Wie waers mit: Diese Menge enthaelt auch negative Zahlen.
positiven, sondern auch aus negativen Zahlen. Ebenfalls könnte man diese Zahlenreihe bis in die Unendlichkeit fortsetzen, %ich mag diese Formulierung nicht
jedoch in diesem Fall in die negative so wie auch in die positive Richtung. Auch hier gibt es wieder eine weitere Untergruppe %eine Untergruppe ist eine spezielle mathematische Struktur, und das wuerd in dem Fall sogar stimmen, aber das muesstest definieren. Das richtige Wort hier ist Untermenge
:\
\begin{itemize}
\item $\mathbb{Z}^*$ - sind alle ganzen Zahlen ohne der Zahl 0
\end{itemize}

\textbf{Rationale Zahlen} - %$\mathbb{Q}$ - \{-7/5, -2/4, 1/2, 2/4, 0,75, ..., n/n\}. 
%n/n ist eins.
$\mathbb{Q} = \{\frac{m}{n} | m \in \mathbb{Z}, n \in \mathbb{N}^*\}$
Dies sind jene Zahlen, welche sich durch einen Bruch darstellen lassen und deren Zähler und Nenner ganze Zahlen sind. %Problem bei 0/0, beides ganze Zahlen
Auch 0,758 lässt sich als Bruch darstellen $\rightarrow{0,758 = 758/1000}$. Diese Zahlen können positiv als aus negativ sein und können unter anderem eine unendliche Anzahl an Nachkommastellen aufweisen.

\textbf{Irrationale Zahlen} - $\mathbb{I}$ - Es handelt sich hier um Zahlen, welche nicht mittels eines Buches dargestellt werden können. Zum Beispiel ist die Kreiszahl $\pi$ (Umfang / Durchmesser) 3,141592654… eine irrationale Zahl. Man erkennt, dass diese Art von Zahlen niemals periodisch werden. Ebenso gilt, dass alle Wurzeln natürlicher Zahlen, welche keine natürlichen Zahlen ergeben, ebenfalls irrational sind ($\sqrt{5}$). %andere moegliche Formulierung: natuerliche Zahlen, die keine Quadratzahlen sind. Wurzelziehen hast du nirgendwo definiert btw, du koenntest zB hinschreiben die positive Loesung der Gleichung x^2=n

\textbf{Reelle Zahlen} - $\mathbb{R}$. Diese setzen sich aus den rationalen und den irrationalen Zahlen zusammen. Sprich, $\mathbb{R}$ sind alle Zahlen \autocite{vgl. Engel 2017: 7-9}. %NEIN. Es gibt zB Komplexe Zahlen.

Es gibt noch weitere Zahlen wie die imaginären Zahlen oder die komplexen Zahlen. Diese sind für die Arbeit nicht relevant und werden daher nicht näher ausgeführt.

\subsubsection{Zusammengesetzte Zahlen}
Um ein noch besseres Bild der Primzahlen verschaffen zu können ist es essentiell zu verstehen was der Begriff \textbf{"{zusammengesetzte Zahlen}"} bedeutet. Diese Zahlen lassen sich aus 2 natuerlichen Zahlen multiplikativ kombinieren. Wie genau dies aussieht, wird im späteren Verlauf dieser Arbeit, durch die Primfaktorzerlegung dargestellt und vermittelt ein gutes Bild wie Zahlen zusammengesetzt werden können. Definitionsgemäß werden diese Zahlen folgender Maßen beschrieben:\\
\hspace*{10mm}\textbf{Sofern eine Zahl $n > 1$ existiert, welche keine Primzahl ist, bezeichnet man diese als zusammengesetzte Zahl}. (vgl. Rempe \& Waldecker 2009: 13).\\

\subsection{Teilbarkeit und Vielfaches}
Um uns dieser Thematik widmen zu können müssen wir ein paar wenige Parameter im Vorfeld definieren.\newline
Wir benötigen hierfür die Variablen $k$, $n$, und $m$ und definieren wie folgt:\newline
Es seien $n$, $m$ und $k \in \mathbb{Z}$. Wir beschreiben $k$ als einen \textbf{Teiler} von $n$, und die Variable $n$ als ein \textbf{Vielfaches} von $k$, wenn es eine ganze Zahl $m$ gibt, sodass die Multiplikation von $m$ und $k$ eine ganze Zahl ergibt. Wir schreiben $k\cdot{m} = n$. Diese Relation wird dann als $k \mid n$ beschrieben.\newline
Als Beispiel können wir hier $4 \mid 8$ anführen, denn $4 \cdot 2 = 8$. Umgekehrt können wir festhalten, dass $5 \cdot 2 \neq 8$. 
Wir schreiben in einem solchen Fall $5 \nmid 8$.\newline
Sofern $k \in \mathbb{Z}$ ist, dann besitzt dieser Therm auf jeden Fall 1, -1, $k$ und $-k$ als Teiler. Demnach ist hier sehr gut zu erkennen, dass jede ganze Zahl mindestens 4 Teiler hat
\begin{center}
$k = k \cdot 1 = 1 \cdot k$\\
$-k = -k \cdot 1 = -1 \cdot k$.
\end{center}
Besitzt nun eine ganze Zahl ausschließlich diese 4 Teiler werden diese als \textbf{triviale Teiler} bezeichnet. Hat hingegen eine ganze Zahl mehrere Teiler werden diese als \textbf{nicht-triviale Teiler} betitelt. In diesem Zusammenhang ist es gut ersichtlich, das Primzahlen ausschließlich aus triviale Teiler bestehen, da sie nur durch 1 und sich selbst teilbar sind. Hierzu werde ich im Verlauf dieser Arbeit aber noch genauer eingehen.

\subsubsection{Teilbarkeitsregeln}
Seien $k, n, m \in \mathbb{Z}$. Wenn $n$ und $m$ von $k$ geteilt werden, dann auch $n$ + $m$,\newline
$n$ - $m$ und $n\cdot m$. Jeder Teiler von $k$ teilt somit auch $k \cdot x$, sofern $x$ ebenfalls $\in \mathbb{Z}$ ist. Daraus ergibt sich: $k \mid n$, $n \mid m$ und $k \mid m$.\\
Um dies ein wenig zu verdeutlichen führe ich ein kurzes Beispiel mit den selben angeführten Variablen an: Wieder nehmen wir an, dass $k$ ein Teiler von $n$ und $m$ ist. Nun fügen wir nach der Teilbarkeitsregel die ganzen Zahlen $x$ und $y$ hinzu, welche aus $k \cdot x = n$ und $k \cdot y = m$ entstehen. Folglich muss gelten:
\begin{center}
$n + m = k \cdot x + k \cdot y = k \cdot (n + m)$\\
$n - m = k \cdot x - k \cdot y = k \cdot (n - m)$\\
$n \cdot m = (kx) \cdot (ky) = k \cdot (x \cdot k \cdot y)$.
\end{center}
Da nun $n + m$, $n - m$ und $x \cdot k \cdot y$ ganze Zahlen sind, ist $k$ nach der Teilbarkeitsregel ein Teiler von $n + m$, $n - m$ und $n \cdot m$ (vgl. Rempe \& Waldecker 2009: 13-15).
\subsubsection{Definition - Größter gemeinsamer Teiler - ggT}
Den größte gemeinsamen Teiler - ggT für $n$ und $m$ für zwei ganze Zahlen $n, m\in \mathbb{Z}$ nennt man
\begin{center}
ggT($n, m$) := max \{$k \mid k$ teilt $n$ und $m\}$.
\end{center}
\subsubsection{Definition - Kleinste gemeinsame Vielfache - kgV}
Das kleinste gemeinsame Vielfache - kgV für $n$ und $m$ für zwei ganze Zahlen $n, m\in \mathbb{Z}$ nennt man
\begin{center}
kgV($n, m$) := min \{$k \mid n$ und $m$ teilen $k\}$.
\end{center}
\subsubsection{Definition - Teilerfremd}
Zwei ganze Zahlen $n, m$ werden als \textbf{teilerfremd} bezeichnet, wenn ggT($n, m$) = 1 ist. Dies bedeutet, dass $n$ und $m$ außer 1 keinen weiteren positiven gemeinsamen Teiler besitzen.

\subsection{Der Euklidische Algorithmus}
Wir haben nun einiges über den ggT in Erfahrung bringen können und ich möchte auch mit der Thematik weiter machen und auf den Euklidischen Algorithmus näher eingehen. Dieser beschäftigt sich ebenfalls mit der Findung des ggTs.\newline
Weiterführend zur Teilbarkeitsregel 1.3.1 besteht ein linearer Zusammenhang des ggT, wenn $x, y \in \mathbb{Z} \neq 0$, dann sind auch $a, b \in \mathbb{Z}$ folglich
\begin{center}
ggT($x, y$) = $ax + by$.
\end{center}
Die in (1.3) angeführte Situation ist für kleinere Zahlen perfekt anwendbar. Auch könnte uns die Primfaktorzerlegung, welche im weiteren Verlauf dieser Arbeit behandelt wird, uns ans Ziel führen den ggT zu finden. Jedoch haben wir bisher nur theoretisch mit variablen gerechnet. Es wird sehr schnell klar, dass die oben angewandte Methode an seine Grenzen stößt sobald es um größere Zahlen geht. Genau hier kommt der Euklidische Algorithmus zum Einsatz. Dieser Algorithmus ist im Werk $EUKLIDS$ Elementen Buch VII Satz 2 zu finden. Er ist mit Sicherheit einer der ältersten Vorgänge der Zahlentheorie und ist bis heute noch als das schnellste Verfahren zur Ermittlung des ggTs bekannt. Er kommt speziell bei großen Zahlen zur Anwendung wo die Primfaktorzerlegung an ihre Grenzen stößt und auch kein ggT mit den in (1.3) angeführten Methoden ermittelt werden kann (vgl. Crandall \& Pomerance 2005: 83-84).\newline
Der Grundgedanke ist jener, dass man aus den bereits bekannten Variablen $a$ und $b$ neue Zahlen generiert. Man hat aber darauf zu achten, dass diese dieselben gemeinsamen Teiler haben.

\begin{quote}
\small
\textbf{Hilfssatz (Zahlenpaare mit den selben gemeinsamen Teilern).} Es seien $a, b, m \in \mathbb{Z}$ beliebig. Dann ist jeder gemeinsame Teiler von $a$ und $b$ auch ein gemeinsamer Teiler von $a$ und $c := b + m \cdot a$; umgekehrt ist jeder gemeinsame Teiler von $a$ und $c$ ein gemeinsamer Teiler von $a$ und $b$. Insbesondere gilt ggT($a, b$) = ggT($a, b + m \cdot a$).\newline
(Rempe \& Waldecker 2009: 17)\newpage
\end{quote}
\textbf{Der Euklidische Algorithmus kann nun wie folgt verstanden werden:}\newline
Gegeben ist $a, b \in \mathbb{N}$ mit $a \ge b \ge 1$. Weiters bestimmen wir das $a_0 := a$ und $a_1 := b$. Hiermit werden wir Divisionsketten mit Rest bilden.
\begin{tabbing}
xxxxx\=xxxxxxxxxxxxxxxxxxxxx\=xxxxxxxx\=xxxxxxxxxxxxxxx\=xxxxxxxxxxxxxx\kill
\> $a_0 = c_1 \cdot a_1 + a_2$             \> mit \> $c_1, a_2 \in \mathbb{Z}$,     \> $0 \le a_2 < a_1$,  \\
\> $a_1 = c_2 \cdot a_2 + a_3$             \> mit \> $c_2, a_3 \in \mathbb{Z}$,     \> $0 \le a_3 < a_2$,  \\
\> \ \ \ \ \ \ \ \ \ \ \vdots              \>     \>\ \ \ \ \ \vdots                \>\ \ \ \ \ \ \ \vdots \\
\> $a_{n-2} = c_{n-1} \cdot a_{n-1} + a_n$ \> mit \> $c_{n-1}, a_n \in \mathbb{Z}$, \> $0 \le a_n < a_{n-1}$.
\end{tabbing}
Nun stellen $c_1, c_2,$ ... die Quotienten und $a_2, a_3,$ ... die Reste dar. Daraus ergibt sich ein erster Index $k, 1 \le k \le b$. Somit gilt: $a_k > 0$, $a_{k+1} = 0$.\newline
Im Ergebnis erkennt man, dass die Zahl $a_k$ der ggT von $a$ und $b$ ist.

Um dies zu veranschaulichen folgen nun 2 Beispiele mit etwas größeren Zahlen:
\begin{align}
a = 138.777 &,\ b = 16.855        & a = 2.250 &,\ b = 765 \\
138.777 &= 8 \cdot 16.855 + 3937 & 2.250 &= 2 \cdot 765 + 720 \\
16.855  &= 4 \cdot 3.937 + 1.107 & 765   &= 1 \cdot 720 + 45 \\
3.937   &= 3 \cdot 1.107 + 616   & 720   &= 16 \cdot 45 \\
1.107   &= 1 \cdot 616 + 491     &  \\
616     &= 1 \cdot 491 + 125     &  \\
491     &= 3 \cdot 125 + 116     &  \\
125     &= 1 \cdot 116 + 9       &  \\
116     &= 12 \cdot 9 + 8        &  \\
9       &= 1 \cdot 8 + 1         &  \\
8       &= 8 \cdot 1
\end{align}

\begin{tabbing}
xxxxx\=xxxxxxxxxxxxxxxxxxxxxxxxxxxxxxx\=xxxxx\kill
\> ggT $(138.777, 16.855) = a_{10} = 1$ \> ggT $(2.250, 765) = a_3 = 45$
\end{tabbing}

Auf der linken Seite der beiden Beispiele erhält man im Ergebnis als\\
ggT ($138.777, 16.855$) nach der zehnten Division die Zahl 1. Auf der rechten Seite ist der ggT ($2.250, 765$) die Zahl 45.
Betrachtet man nun diese Divisionkette von oben nach unten so kann man schlussfolgern, dass für jeden gemeinsamen Teiler $x$ von $a_0$ und $a_1$ hintereinander $x \mid a_0$, $x \mid a_1$, $x \mid a_2$, ..., $x \mid a_k$. Daher erfüllt $a_k$ die Definition ein ggT zu sein $\to$ $a_k$ = ggT($a_0,a_1$) = ggT($a, b$) (vgl. Remmert \& Ullrich 1995: 58-59).

\section{Modulare Arithmetik}
Ich werde bei den Grundlagen dieser Thematik genauer in die Tiefe gehen, um die Zusammenhänge im späteren Verlauf leichter fassen zu können. Durch gewisse Eigenschaften, welche die modulare Arithmetik mit sich bringt, konnten einige höherrangige Primzahltests entwickelt und implementiert werden. Von besonderen Interesse kann hier der \textit{Kleine Satz von Fermat} genannt werden, welcher im Kapitel 4.3 näher beschrieben wird.

Wichtig ist hier im ersten Schritt zu verstehen, dass die modulare Arithmetik eine differenziere Betrachtungsweise der Additions- bzw. Multiplikationsrechnung darstellt. Für eine leichtere Auffassung ist die beste Methode sich eine Uhr und deren Ziffernblatt vorzustellen. Möchte man zum Beispiel wissen wie spät es in 55 Stunden sein wird wenn es jetzt 16 Uhr ist, wäre eine klassische Addition von $16+55=sage$ 
inkorrekt, da ein klassisches Ziffernblatt nur 12 Zahlen und der Tag bekannterweise nur 24 Stunden hat. Wobei in dieser Betrachtung aber gilt $12=0$. Man kann nun das Ergebnis 71 durch 24 Stunden dividieren und enthält $2,958333333$. Weiters wird der Rest $0,958333333$ nochmals mit 24 multipliziert. Somit gelangt man zur korrekten Lösung, dass es in 55 Stunden \textbf{23 Uhr} sein wird.
Bevor ich dies noch mathematisch darstelle und beweise möchte ich noch grundlegende Bedingungen festlegen:

\begin{itemize}
    \item Es ist bedeutend, dass eine natürliche Zahl $n>1$ beim teilen mit $n$ immer den selben Rest hat.
    \item Weiters bleibt beim teilen durch 2 von ungeraden $\mathbb{Z}$ immer ein Rest von 1.
    \item Beim bilden der Summe zweier geraden Zahlen erhält mal immer eine gerade Zahl $\rightarrow$ $2+2=4$.
    \item Gleich verhält es sich bei ungeraden Zahlen $\rightarrow$ $3+3=6$.
    \item Die Summe einer geraden und ungeraden Zahl ist stets ungerade $\rightarrow$ $2+3=5$.
    \item Das Produkt einer geraden und einer zufälligen $\mathbb{Z}$ ist immer gerade $\rightarrow$ $2 \cdot -3 = -6$.
    \item Das Produkt von zwei ungeraden Zahlen ist stets ungerade $\rightarrow$ $3 \cdot 3 = 9$.
\end{itemize}

Um diese Bedingungen per Definition festzuhalten formulieren wir:\\
\textbf{Kongruenz:} Für $\mathbb{N}$ gilt $n>1$. $a$ und $b$ sind $\mathbb{Z}$, welche beim Teilen mit $n$ den selben Rest haben. Somit erhalten wir $a\equiv b$ (mod $n$). Sprich: \textbf{$a$ ist kongruent zu $b$ modulo $n$}. Die Differenz dieser Zahlen ist ein Vielfaches von $n$, wenn $a-b$ ein Vielfaches von $n$ ist.


\section{Primzahlen}
\subsection{Die Anfänge der Primzahlen}
Die Materie der Primzahlen ist ein sehr komplexes und bis heute noch immer nicht ganz verständliches Zahlensystem. Es ist zwar bekannt, dass es Primzahlen gibt, aber man glaubt, dass es unendlich viele Primzahlen geben muss. Äußerst erstaunlich ist hingegen die frühe Entdeckung über die Existenz der Primzahlen. Noch vor Christi Geburt wurden diese Sonderlinge von einem gewissen Herrn \textit{Euklid} oder \textit{Euklides} entdeckt (Remmert und Ullrich 1995: 22). Dieser Mann war ein griechischer Gelehrter und der berühmteste Mathematiker der Antike. Er wurde etwa 60 Jahre alt und starb circa 265 v. Chr. in Alexandria, Ägypten. Sein Werk "Die Elemente"\ beeinflusste zwei Jahrtausende die Welt der Mathematiker und ist neben der Bibel als das weit verbreitetste Werk der Weltliteratur bekannt. Obwohl er für so lange Zeit der Welt der Mathematik prägte ist sehr wenig über sein Leben und seine Herkunft bekannt. Schriften zufolge soll er die platonische Akademie in Athen besucht haben. Über die Zeit haben sich drei unterschiedliche Theorien über sein Leben entwickelt. Es stellen sich die Fragen ob Euklides\
\begin{itemize}
\item tatsächlich eine historische Person ist und er seine Werke selbst verfasst hat, oder
\item der Kopf einer Gruppe gewesen sei, welche in seinem Namen die Errungenschaften und Erkenntnisse veröffentlichten, selbst nach seinem Tod, oder
\item er überhaupt gelebt hat? Spezialisten glauben, dass damalige Mathematiker den Namen eines Herrn \textit{Euklid von Megara} verwendeten und deren Werke unter dessen Namen veröffentlichten. Dieser Mann lebte aber etwa 100 Jahre vor der Geburt Euklides.
\end{itemize}
Welche dieser Theorien nun stimmt kann ich nicht beurteilen ist auch nicht Bestandteil dieser Arbeit. Fakt ist jedoch, dass diese Werke maßgebend sind und über 2000 Jahre deren Richtigkeit nicht angezweifelt wurden. Das Werk  "Die Elemente"\ besteht aus 13 Büchern. Unter anderem handelt das siebente Buch von der Zahlentheorie und beinhaltet den \textit{Euklidischen Algorithmus}, welche zur Bestimmung des größten gemeinsamen Teilers (ggT) zweier Zahlen dient (Biografie von Euklid).\\
Genau hier möchte ich einen sanften Übergang in die Zahlentheorie einleiten. Diese beschäftigt sich mit den Eigenschaften von ganzen Zahlen und zerlegt unter anderem eine Zahl in Primfaktoren. Es ist derzeit immer noch ein Problem Primzahlen überhaupt zu erkennen. In dieser Disziplin der Mathematik gibt es eine Vielzahl an Problemen, welche sich leicht formulieren lassen. Diese aber zu lösen stellt sich bis heute noch als außerordentlich schwierig dar (vgl. Pomerance 1983: 136).

\subsection{Primzahlen per Definition}
\textbf{Primzahlen} - $\mathbb{P}$ - \{2, 3, 5, 7, 11, 13, 17, 19, 23, 29, 31, 37, 41, ..., pn\}.\\
Die Definitionen von Primzahlen lauten wie folgt:\\
\hspace*{5mm}\textbf{Primzahlen sind natürliche Zahlen, die exakt zwei Teiler besitzen.}\\
\hspace*{5mm}\textbf{Primzahlen sind natürliche Zahlen, welche genau durch 2 verschiedene natürliche Zahlen geteilt werden können.}

Nach diesen Definitionen ist eine Zahl dann eine Primzahl wenn diese durch 1 und sich selbst ohne Rest teilbar ist.
\paragraph{Exkurs zu den Zahlen 0 und 1:}Die Zahlen 0 und 1 sind keine Primzahlen!\\
Die Zahl 0 kann nicht durch sich selbst geteilt werden und daher spricht sie gegen diese Definition. Die Berechnung 0:0 ist nicht erlaubt.\\
Die Zahl 1 war unter Mathematikern lange als Primzahl anerkannt, da sie der oben genannten Definition durchaus entspricht. Doch wurde sie aufgrund von mehreren Problemen mit der Zeit aus der Reihe der Primzahlen entfernt. Zum einen erfüllt sie zwar das Kriterium der Teilbarkeit durch 1 und sich selbst, doch hat die Zahl 1 nur einen Teiler, wohingegen alle anderen Primzahlen zwei Teiler besitzen. Auch führt diese Zahl bei der Primfaktorzerlegung zu Problemen, welche weiter unten angeführt werden.

\subsection{Welche Primzahlen gibt es?}
Eine Aufzählung der unterschiedlichsten Primzahlen:
\begin{itemize}
\item [a]\underline{Mersenne-Primzahlen:} Diese haben die Form $2^{n}-1$ (aber nicht jede Zahl für n liefert  eine Mersenne-Primzahl, so ergibt z.B. n = 4 die Zahl 15, die ja keine Primzahl ist).
Die ersten Mersenne-Primzahlen lauten: 3, 7, 31, 127, 8.191, 131.071, 524.287, 2.147.483.647, …
\item [b]\underline{Megaprimzahlen:} Diese sind Primzahlen mit mindestens 1.000.000 Stellen in ihrer dezimalen Darstellung.
Anmerkung: Die größte im Januar 2017 bekannte Primzahl hat 22.338.618 Stellen und kann so berechnet werden: $2^{74.207.281}-1$.
\item [c]Fermat-Primzahlen
\item [d]Wilson-Primzahlen
\item [e]Kubanische-Primzahlen
\item [f]Lucas-Primzahlen
\item [g]Sern-Primzahlen
\item [h]Cullen-Primzahlen
\item [i]Wagstaff-Primzahlen
\item [j]Proth-Primzahlen
\item [k]Germain-Primzahlen
\item [l]Truncatable-Primzahlen
\item [m]Repuni-Primzahlen
\item [n]Stare-Primzahlen
\item [o]Zyklische-Primzahlen
\item [p]Permutierbare-Primzahlen
\end{itemize}
Wie oben zu erkennen ist gibt es unterschiedlichste Formen von Primzahlen und alle werden unterschiedlich berechnet. Genauso gibt es zu jeder Primzahl auch verschiedene Testungen um festzustellen ob das Ergebnis tatsächlich eine Primzahl darstellt (vgl. Engel 2017: 86-90).

\subsection{Das Sieben von Zahlen}
Das Sieben von Zahlen kann eine sehr effektive Funktion für einige Berechnungen sein. Z.B. kann mit dem Sieben heraus gefunden werden ob eine Zahl eine Primzahl ist oder auch für die Faktorisierung verwendet werden. Ich gehe hier näher auf das \textbf{Sieb des Eratosthenes} ein. Dieses sticht mit seiner Einfachheit heraus. Selbst ein Kind, welches ein gewisses Maß an mathematischen Wissen besitzt kann diesen Test sehr leicht durchführen. Er wurde im 3. Jahrhundert v. Chr. mit dem Grundgedanken entwickelt um für eine gewisse Bandbreite an Zahlen alle Primzahlen heraus zu filtern. Anhand des folgenden Beispiels werde ich dies näher erörtern:\\
Für die unten stehenden Zahlen 1 bis $N$ ($1-151$) in der Tabelle möchte ich nun alle enthaltenen Primzahlen herausfinden. Wie schon anfangs des Kapitels erwähnt wird die Zahl 1 nicht als Primzahl gewertet und daher wird diese in der Tabelle weggelassen und die Zahlenfolge beginnt mit der Zahl 2. Um die Arbeit weiter zu erleichtern werden alle ungeraden und geraden Zahlen untereinander aufgeschrieben.\\

\begin{center}\begin{tabular}{|c|c|c|c|c|c|c|c|c|c|}
\hline
   & 2 & 3 & \st{ 4 } & 5 & \st{ 6 } & 7 & \st{ 8 } & \st{ 9 } & \st{ 10 } \\
11 & \st{ 12 } & 13 & \st{ 14 } & \st{ 15 } & \st{ 16 } & 17 & \st{ 18 } & 19 & \st{ 20 } \\
\st{ 21 } & \st{ 22 } & 23 & \st{ 24 } & \st{ 25 } & \st{ 26 } & \st{ 27 } & \st{ 28 } & 29 & \st{ 30 } \\
31 & \st{ 32 } & \st{ 33 } & \st{ 34 } & \st{ 35 } & \st{ 36 } & 37 & \st{ 38 } & \st{ 39 } & \st{ 40 } \\
41 & \st{ 42 } & 43 & \st{ 44 } & \st{ 45 } & \st{ 46 } & 47 & \st{ 48 } & \st{ 49 } & \st{ 50 } \\
\st{ 51 } & \st{ 52 } & 53 & \st{ 54 } & \st{ 55 } & \st{ 56 } & \st{ 57 } & \st{ 58 } & 59 & \st{ 60 } \\
61 & \st{ 62 } & \st{ 63 } & \st{ 64 } & \st{ 65 } & \st{ 66 } & 67 & \st{ 68 } & \st{ 69 } & \st{ 70 } \\
71 & \st{ 72 } & 73 & \st{ 74 } & \st{ 75 } & \st{ 76 } & \st{ 77 } & \st{ 78 } & 79 & \st{ 80 } \\
\st{ 81 } & \st{ 82 } & 83 & \st{ 84 } & \st{ 85 } & \st{ 86 } & \st{ 87 } & \st{ 88 } & 89 & \st{ 90 } \\
\st{ 91 } & \st{ 92 } & \st{ 93 } & \st{ 94 } & \st{ 95 } & \st{ 96 } & 97 & \st{ 98 } & \st{ 99 } & \st{ 100 } \\
101 & \st{ 102 } & 103 & \st{ 104 } & \st{ 105 } & \st{ 106 } & 107 & \st{ 108 } & 109 & \st{ 110 } \\
\st{ 111 } & \st{ 112 } & 113 & \st{ 114 } & \st{ 115 } & \st{ 116 } & \st{ 117 } & \st{ 118 } & \st{ 119 } & \st{ 120 } \\
\st{ 121 } & \st{ 122 } & \st{ 123 } & \st{ 124 } & \st{ 125 } & \st{ 126 } & 127 & \st{ 128 } & \st{ 129 } & \st{ 130 } \\
131 & \st{ 132 } & \st{ 133 } & \st{ 134 } & \st{ 135 } & \st{ 136 } & 137 & \st{ 138 } & 139 & \st{ 140 } \\
\st{ 141 } & \st{ 142 } & \st{ 143 } & \st{ 144 } & \st{ 145 } & \st{ 146 } & \st{ 147 } & \st{ 148 } & 149 & \st{ 150 } \\
151 & & & & & & & & & \\
\hline
\end{tabular}\end{center}

Nun gehen wir wie folgt vor:
\begin{enumerate}
    \item Da 2 prim ist können wir alle Vielfachen $>2$, beginnenend bei 4 aus der Liste streichen.
    \item Nun sehen wir uns die nächste Zahl 3 an. Diese ist auch prim und somit können wir wie zuvor in 1. alle Vielfachen $>3$, beginnend mit 6, streichen.
    \item Da die Zahl 4 schon im ersten Schritt gestrichen wurde gehen wir direkt zur Zahl 5 über. Auch hier erkennen wir, dass diese prim ist und können nun wieder alle Vielfachen $>5$, beginnend mit 10, streichen.
    \item Dies setzen wir nun solange fort bis wir uns alle noch nicht gestrichenen Zahlen in der Tabelle angesehen haben und alle Vielfachen der kleinsten noch übrig gebliebenen Zahlen $p$, welche $>p$ sind, gestrichen wurden. Man hat diese Folge abgeschlossen wenn $p^2 > 151$. 
\end{enumerate}
Mit dem Sieb des Eratosthenes werden in diesem Beispiel alle Vielfache von $2, 3, 5, ... < \sqrt{151}$ ausgesiebt bis wir bei $\lceil151\rceil$ angelangt sind. Der Vorteil liegt darin, dass bereits gestrichene Zahlen nicht weiter berücksichtigt werden müssen, da diese durch einen echten Teiler teilbar und somit nicht prim sind. Nicht gestrichene Zahlen welche zwischen 1 und 151 liegen sind somit Primzahlen. Unter Berücksichtigung von einigen Optimierungen, auf welche ich hier nicht eingehe, kann dieser Sieb mathematisch wie folgt dargestellt werden:
\[\sum_{p\le N} \frac{N}p = N ln ln N + O(N)\]
(vgl. Crandall \& Pomerance 2005: 121)

Um dieses Beispiel abzuschließen ergibt sich durch das Sieb folgende Reihe an Primzahlen $\mathbb{P}$: %$\sage{prime_range(152)}$.\\
Für kleine Zahlen ist dieses Siebverfahren durchaus geeignet, doch auch hier lässt sich erkennen, dass es für große Zahlen nicht praktikabel ist.

Es gibt natürlich weitere Primzahltestungen:
\begin{itemize}
    \item Die einfachste von allen ist die \textit{Probedivision}. Hier wird durch eine simple Division versucht ob eine Zahl durch eine weitere Zahl, außer 1 und sich selbst, teilbar ist.
    \item Eine verbesserte Version des Siebes von Eratosthenes ist das \textit{Sieb von Atkin}. Dieses hat einen schnelleren und moderneren Algorithmus um Primzahlen zu benennen.
    \item Auf dem \textbf{Kleinen Satz von Fermat} beruht die \textit{fermatsche Primzahltestung}. Auf diesen Satz beruht aber auch unter anderem der
    \item \textit{Lucas-Lehmer-Test}, welcher zur Prüfung von Mersenne-Primzahlen dient. Auf diesen Test werde ich im Kapitel 4.5 Mersenne-Primzahlen genauer eingehen.
    \item Eine substanzielle Modifizierung vom kleinen fermatschen Satz ist der \textit{APRCL-Test}. Dieser wurde 1980 von 5 Mathematikern ent- und weiterentwickelt.
\end{itemize}

\subsection{Der Kleine Satz von Fermat}


Um die Existenz von unendlich vielen Primzahlen zu beweisen wurden unterschiedliche Theorien und mathematische Formeln aufgestellt. Der womöglich wichtigste Beweis “Beweis von Euklid” wird in dieser Arbeit angeführt. Es gibt noch mehrere Beweise doch können diese hier nicht alle besprochen werden und würde den Rahmen sprengen.

\subsection{Beweise für die Existenz von Primzahlen}
Die folgende Zahlenreihe kann vermutlich bis ins unendliche weitergeführt werden \{2, 3, 5, 7, 11, 13, 17, 19, 23, 29, 31, 37, 41, …, pn\}. So ist die Zahl 5 nur durch 1 und sich selbst (5) teilbar. Die Zahl 1 ist keine Primzahl, da diese nur einen Teiler besitzt (vgl. Engel 2017: 86). Die Zahl 8 ist keine Primzahl, da sie durch 1, 2, 4 und sich selbst teilbar ist. Generell können alle geraden Zahlen niemals eine Primzahl sein, da jede dieser Zahlen durch 1, sich selbst und mindestens einer weiteren geraden Zahl teilbar ist. Eine Ausnahme ist hier die Zahl (2). Wie man nun erkennt werden Primzahlen als Teilmenge der ganzen Zahlen dargestellt. Es gibt so einige Sätze, welche man für den Beweis einer Primzahl anwenden kann. Jedoch wird die wichtigste und einfachste Form der “Fundamentalsatz der Arithmetik” sein. Dieser besagt, dass jede natürliche Zahl, welche selbst keine Primzahl und die > 1 ist, als Produkt von Primzahlen dargestellt werden kann. Als Beispiele können genannt werden:\
\begin{center}$48 = 24\times{2} = 12\times{2}\times{2} = 6\times{2}\times{2}\times{2} = 3\times{2}\times{2}\times{2}\times{2}$\\
$55 = 11\times5$\\
$153 = 51\times{3} = 17\times{3}\times{3}$\end{center}
Dieses Vorgehen der Aufteilung in die kleinsten Bestandteile einer Zahl nennt man \textbf{Primfaktorzerlegung}. Bei dieser Zerlegung wird die jeweilige Zahl durch die kleinst mögliche Primzahl geteilt bis im Ergebnis kein Rest über bleibt. Mit dem Ergebnis kann weiter gerechnet und dividiert werden bis schlussendlich keine weitere Teilung der Zahl mehr möglich ist. Hier erkennt man, dass die Produkte eindeutig sind. Es gibt keine weitere Zerlegungen der Zahlen 48, 55 oder 153 (vgl. Beweis, dass es unendlich viele Primzahlen gibt). 
\paragraph{Exkurs zur Faktorisierung:}Da hier die Primfaktorzerlegung in Kürze angesprochen wurde möchte ich einen kurzen Ausflug zur Faktorisierung ermöglichen. Im späteren Verlauf dieser Arbeit werde ich noch näher darauf eingehen und auch das Faktorisierungsproblem aufgreifen.\\
Die Zerlegung einer kleinen Zahl wie zum Beispiel der Zahl 153 stellt kein großes Problem dar und kann einfach mit einem Taschenrechner von statten gehen. Für große Zahlen jedoch bedarf es ausgeklügelte Programme und Algorithmen, welche allein mit einem Taschenrechner nicht mehr zu bewerkstelligen sind.\\
Beispiele für die Primfaktorzerlegung einiger Mersenne Zahlen, welche noch vor der Erfindung von Computern herausgefunden wurden:\
\begin{center}
$M_{59} = 2^{59}-1 = 179.951\times3.203.431.780.337$ $\rightarrow$ 1869 von Landry;\\
$M_{67} = 2^{67}-1 = 193.707.721\times761.838.257.287$ $\rightarrow$ 1903 von Cole;\\
$M_{73} = 2^{73}-1 = 439\times2.298.041\times9.361.973.132.609$ $\rightarrow$ 1856 von Clausen.\\
\end{center}
Weitere Beispiele der Faktorisierung aus der jüngeren Zeit:\
\begin{center}
$M_{113} = 2^{113}-1 = 3.391\times23.279\times65.993\times1.868.569\times1.066.818.132.868.207$ $\rightarrow$ 1856 von Reuschle den kleinsten Faktor \& 1947 die restlichen Fakroten von Lehmer;\\
$M_{193} = 2^{193}-1 = 13.821.503\times61.654.440.233.248.340.616.559\times14.732.265.321.145.317.331.353.282.383$ $\rightarrow$ 1983 von Naur;\\
$M_{257} = 2^{257}-1 = 535.006.138.814.359\times1.155.685.395.246.619.182.673.033\times374.550.598.501.810.936.581.776.630.096.313.181.393$ $\rightarrow$ 1979 von Penk den ersten Faktor \& 1980 von Baille, der die anderen beiden Faktoren fand\end{center} (vgl. Ribenboim: 2006).

\subsubsection{Beweis von Euklid}
Als einfachster Beweis kann hier der Beweis von Euklid genannt werden. Gehen wir davon aus, dass wir den Fakt der bis heute herrschenden Auffassung “Es gibt unendlich viele Primzahlen” widerlegen möchten. Demnach formulieren wir hierzu zwei widersprüchliche Aussagen:\\\\
\hspace*{20mm}P = Es gibt unendlich viele Primzahlen\\
\hspace*{20mm}N = Es gibt endlich viele Primzahlen\\
\\Da bei diesen Aussagen das Prinzip des Widerspruchs herrscht kann nur eine der beiden Aussagen korrekt sein. Wir wollen nun den Beweis antreten, dass N richtig und P inkorrekt ist. Demnach würde dies bedeuten, dass es eine höchste Primzahl “$p_r$” geben müsste. Nehmen wir nun die Reihe der oben angeführten Primzahlen:\\
\begin{center}
$2, 3, 5, 7, 11, 13, 17, 19, 23, 29, 31, 37, 41, $…$, p_r$
\end{center}\
\\Als nächsten Schritt definieren wir eine neue Zahl $x$, nehmen die Zahlenreihe, multiplizieren diese unter einander in gleicher Reihenfolge und addieren am Ende der Reihe eine 1 hinzu. Daraus würde sich folgendes Bild ergeben:\\
\begin{center}
$x = 2\times3\times5\times7\times11\times13\times17\times19\times23\times29\times31\times37\times41\times$…$\times$ $p_r$ + 1
\end{center}\
\\Die Addition von 1 am Ende der Multiplikation ist äußerst wichtig, da dadurch das Ergebnis $x$ nicht mehr exakt durch eine Primzahl aus der endlichen Reihe geteilt werden kann.\\
Zur Auflösung dieser Zahlenreihe gibt es zwei mögliche Ergebnisse:
\begin{itemize}
\item[1.] $x$ ist selbst eine Primzahl, oder
\item[2.] $x$ ist teilbar durch eine Primzahl die größer ist als $p_r$.
\end{itemize}
\textbf{Zu 1.:} Sollte $x$ tatsächlich eine Primzahl sein, dann wäre hiermit schon bewiesen, dass die Aussage N nicht stimmen kann. Da am Ende der Zahlenreihe $p_r$ + 1 gerechnet wird, muss $x$ > $p_r$ sein.\\\\
\textbf{Zu 2.:} Sehen wir uns nochmals den oben beschriebenen Fundamentalsatz der Arithmetik und deren Primfaktorzerlegung an. Dieser Satz besagt, dass sich jede Zahl als Produkt von Primzahlen darstellen lässt. Da sich $x$ durch die Addition von 1 nicht durch eine bekannte Primzahl aus der Zahlenreihe teilen lässt muss es doch im Umkehrschluss eine Primzahl geben, welche größer als $p_r$ ist, um $x$ teilbar zu machen. Auch hier gilt wiederum N als widerlegt.\\
\\Da nun N als die falsche Aussage identifiziert wurde kann nur P mit der Aussage, dass es eine unendliche Anzahl von Primzahl gibt, Gültigkeit haben (vgl. Ribenboim 2006: 3).\\
\\Dieser Beweis von Euklid ist an sich einer der einfachsten, doch aufgrund des Nichtwissens wie die neue Primzahl $p_r$ beschaffen ist, welche in jedem Schritt erzeugt wird, wirft dieser Beweis weitere Fragen auf. Zum einen lässt sich nur erkennen, dass die neue Primzahl $p_r$ in ihrer Größe höchstens gleich
\\*$x$ = $p_1p_2...p_n$ + 1 ist. Jedoch kann es durchaus sein, dass die Zahl $x$ für manche Indizes $n$ selbst eine Primzahl und für andere $n$ aber teilbar ist.\\
\\Im Buch “Die Welt der Primzahlen” von Paulo Ribenboim findet sich wohl die kürzeste Version von Euklids Beweis und unterstreicht dessen Richtigkeit von der Anzahl unendlicher Primzahlen:\\
\begin{quote}
\small
\textbf{Beweis von Kummer.} Angenommen, es gäbe nur endlich viele Primzahlen $p_1 < p_2 <...< p_r$. Es sei $N = p_1p_2...p_r$, wobei N > 2. Die Zahl N - 1, die wie alle natürlichen Zahlen ein Produkt von Primzahlen ist, muss mit N einen gemeinsamen Teiler $p_i$ haben, der dann wiederum die Differenz N - (N - 1) = 1 teilen müsste, was nicht sein kann. (Ribenboim 2006: 4)
\end{quote}

\subsubsection{Bézout's Lemma}
\textbf{Auch hier muss noch über dieses Thema recherchiert werden!}\\

Nun aber einen Blick zu den Mersenne-Primzahlen und eine Erklärung wann eine Zahl eine Mersenne Zahl oder eine Mersenne-Primzahl ist.

\subsection{Mersenne-Primzahlen}
Wie schon anfangs erwähnt haben Mersenne-Primzahlen folgende Form:\\
\begin{center}
$M_n := 2^{n}-1$.
\end{center}\
\\Interessant ist, dass Mersenne-Zahlen mit vollkommenen Zahlen auftreten. Eine natürliche Zahl wird als vollkommen bezeichnet, wenn sie die Summe all ihrer kleineren Teiler ist. Als einfachste Beispiele können hier 2 Zahlenreihen angeführt werden:
\begin{center}
$6 = 1+2+3$\\
$28 = 1+2+4+7+14$
\end{center}
\textbf{\underline{Dies hat schon Euklides gezeigt:} --> Muss ein wenig überarbeitet werden.}
\paragraph{Definition (Satz von Euklides):}Sofern n $\in$ $\mathbb{N}$ ist, dann ist $2^{n}-1$ prim.Weiters ist $2^{n}-1$($2^{n}-1$) eine vollkommene Zahl.\\
\\Diese Definition ist sogar eine Formulierung von vollkommen geraden Zahlen und war lange Zeit unbewiesen. Der dafür notwendige Beweis kam später durch "{Euler}". Da man aufgrund dieser Beweise erkennen kann, dass die Kombination von Marsenne-Zahlen und vollkommenen Zahlen in der Zahlentheorie einen wichtigen Beitrag leisten, können Mersenne-Primzahlen weiter definiert werden:
\paragraph{Definition (Mersenne-Primzahl):}Von einer Mersenne-Primzahl spricht man, wenn die Mersenne-Zahl selbst prim ist.\\
\\Erklärungsgemäß bedeutet dies, dass eine Mersenne-Primzahl nur dann als solches gilt wenn $n$ ebenfalls eine Primzahl ist. Z.B. ist $M_5 = 2^{5}-1 = 31$ eine Mersenne-Primzahl, da $n$, dargestellt als hochgestellte (5), selbst prim ist. Auf den ersten Blick mag es den Anschein haben, dass für eine zusammengesetzte Zahl $n$ auch die Mersenne-Zahl $M_n$ zusammengesetzt sein muss. Dies kann jedoch verneint und kann nicht als Charaktermerkmal angeführt werden. Als Beispiel dient hier $M_{11} = 2^{11}-1 = 2047$. Da $23 \times 89$ die Zahl $2047$ ergibt, und somit mehr als zwei Teiler besitzt und gegen die Definition der Primzahlen verstößt, kann nicht von einer Primzahlen gesprochen werden. Vielmehr wird sie aber als eine zusammengesetzte Zahl bezeichnet (vgl. Rempe \& Waldecker 2009: 168-169).\\
Für interessierte Leser kann hier weiterführend der Lucas-Test erwähnt werden. Dieser Test dient als Beweis und liefert Antworten zum oben angeführten Problem der Primzahlen in Kombination mit zusammengesetzten Zahlen. Die Definition wird wie folgt ohne weitere Erörterung angeführt.
\paragraph{Definition (Lucas-Test):}Wir definieren eine Folge $k_0,k_1,...$ von natürlichen Zahlen rekursiv durch $k_0$ := 4 und $k_{i+1}$ := $k_i^2-2$ für alle $i\geq 0$.
\newline Sei nun $n\geq 3$ prim. Genau dann ist $M_n$ eine Mersenne-Primzahl, wenn $M_n$ ein Teiler ist von $k_{n-2}$. \textbf{Hier muss ich noch ein wenig mehr recherchieren, weil ich mir vorstellen kann, dass das nicht wirklich die Definition vom Lucas-Test ist!!!}\\
\\Es gibt natürlich Unmengen an Forschungsmaterial zu diesen Zahlen, jedoch sind heute noch einige Fragen hierzu unbeantwortet:\\
\\Sind unendlich viele Mersenne-Primzahlen vorhanden?\\
\hspace*{15mm}Derzeit muss die Frage mit "JA"\ beantwortet werden, da es hierzu\\ 
\hspace*{15mm}keine gegenteilige Argumentationen gibt.\\
\\Gibt es unendlich viele zerlegbare Mersenne-Primzahlen?\\
\\Ist jede Mersenne-Primzahl quadratfrei?\\








Wir haben nun einiges über Primzahlen erfahren. Wir haben verschiedene Tests und deren Existenz kennen gelernt. Nun stellt sich aber die Frage, wofür werden Primzahlen und deren Test in den unterschiedlichsten Verfahren denn eigentlich benötigt? Diese Thematik wird mit dem nächsten Abschnitt erklärt. \newpage


\section{Kryptographie}

\subsection{Geschichtliches zur Kryptographie}
Der Grundgedanke von Kryptographie ist eine Nachricht, welche für eine bestimmte Person oder Personengruppe verfasst wurde, zu verschlüsseln und für andere unleserlich zu machen. Archäologen gehen davon aus, dass im Zeitalter der Pharaonen etwa 1900 v. Chr., Ägypter eine Art von Verschlüsselung bei Pharaonengräber verwendeten, da teilweise ungewöhnliche Hieroglyphen benutzt wurden, welche bis heute noch nicht übersetzt werden konnten. Die Geschichte zeigt aber bislang unzählige Versuche der Menschheit, wie diese deren Nachrichten zu kodieren versuchten um nicht einbezogene Gruppen dieser vorzuenthalten.
Um hier wenige Beispiele anzuführen:\\

\textbf{Skytale}, etwa 475 v. Chr., wird des öfteren als erstes verifiziertes und raffiniertes Instrument der Verschlüsselung angeführt. Hierfür bedarf es lediglich zwei  Holzstäbe mit identen Durchmesser und einen Papyrusstreifen. Das Papyrus wurde um einen der Holzstäbe gewickelt und eine Nachricht darauf verfasst. Abgewickelt ergab sich daraus nur ein wirrer Buchstabensalat. Hatte die andere Person, welcher als Empfänger der Nachricht diente, den anderen Holzstab im selben Durchmesser so konnte nur er die Nachricht lesen.\\

Bei der \textbf{Polybius-Tafel}, zirka 190 v. Chr., wurden zum ersten Mal Buchstaben in numerische Zeichen umgewandelt und löste die Skytale ab.\\

Die \textbf{Caesar-Chiffre} wird vermutlich eines der bekanntesten Methoden zur Verschlüsselung sein. Bei dieser Verschlüsselung handelt es sich um eine “monoalphabetische”. Bei dieser Art der Kodierung werden die Buchstaben des Alphabets vertauscht und es entsteht als Endprodukt ein sogenanntes Schlüsselalphabet, welches zur Entschlüsselung von Nachricht unbedingt benötigt wird. Bei der Caesar-Chiffre wurden die Buchstaben des Alphabets einfach um drei Stellen nach rechts verschoben.\\
Man kann sich durchaus leicht vorstellen, dass diese Arten der Verschlüsselung auf Dauer recht unsichere Methoden waren. Sobald man als nicht einbezogene Partei einmal die Grundidee dahinter verstanden hat ist die Dekodierung ein leichtes. Bei den ersten zwei Beispielen müsste man die Nachrichten einfach nur um mehrere Holzstäbe unterschiedlichen Durchmessers wickeln bis man die Nachricht lesen kann. Selbst bei der Anwendung des Schlüsselalphabets ist es irrelevant nach links zu verschieben und, oder die Anzahl der Stellen zu ändern. Das Prinzip bleibt im Grunde gleich.\\

\textbf{ENIGMA}, die damals ausgeklügelste Methode in Form einer Verschlüsselungsmaschine, ist ein weiteres sehr bekanntes Verschlüsselungssystem und wurde von den Deutschen im 2. Weltkrieg eingesetzt. Es wurde eine sogenannte “polyalphabetische” Verschlüsselung verwendet. Bei dieser Art der Verschlüsselung wird nicht nur ein Schlüsselalphabet, wie bei der Caesar-Chiffre, sondern mehrere Alphabete werden kombiniert, zwischen denen immer wieder gewechselt wird. ENIGMA hat automatisch zwischen mehr als 100.000 unterschiedlichen Alphabeten gewechselt. Dieses Gerät wurde erfolgreich über mehrere Jahre angewandt, bis auch diese Methode von einem englischen Mathematiker schlussendlich geknackt wurde (Rempe und Waldecker 2009: 103-105).

\subsection{Kryptographie von Heute}
Durch die geschichtliche Zusammenfassung im vorherigen Abschnitt wurde ein grober aber guter Überblick verschafft wofür die Kryptographie dient. Jedoch möchte ich zusätzlich vermerken, dass sich die Kryptographie generell mit der Entwicklung, Erforschung und Verbesserung von Verschlüsselungsmethoden befasst. Das dahinter steckende Ziel ist es Nachrichten so einfach und schnell wie möglich zu verschlüsseln und für Außenstehende unkenntlich und unbrauchbar zu machen. Als Gegenspieler der Kryptographie wird die Kryptoanalyse bezeichnet. Hier wird versucht die kodierte Nachricht ohne dem Wissen des passenden Schlüssels zu dekodieren und die verwendeten Verschlüsselungssysteme herauszufinden.\\

Bei den oben angeführten Beispielen im Abschnitt 4.1 erkennt man jedoch, dass alle Systeme durch damalige Kryptoanalytiker bezwungen und der Schlüssel herausgefunden wurde. Wo lag also der Fehler bei den unterschiedlichen Verschlüsselungsformen? Alle der vier genannten Beispiele haben denselben Nachteil, sie sind symmetrische Verschlüsselungssysteme. Das bedeutet in diesem Zusammenhang, dass der Schlüssel zum senden und empfangen der Nachricht ident war. Sobald man also den Verschlüsselungsalgorithmus verstanden und den Schlüssel hatte wurde das verwendete Kryptosystem gegenstandslos. Genau dieser Aspekt ist ein großes Sicherheitsrisiko. Um diese Schwachstelle zu umgehen wurde 1975 die sogenannte \textbf{Public-Key-Verschlüsselung} ins Leben gerufen. Um eine Nachricht lesen zu können bedarf es gegenüber der symmetrischen Verschlüsselung nicht nur einen sondern zwei Schlüssel. Jener Schlüssel der zur Kodierung der Nachricht dient wird “public-key” oder “öffentlicher Schlüssel” genannt. Für die Entschlüsselung wird jedoch ein weiterer Schlüssel, der “private-key” oder “privater Schlüssel”, benötigt. Beide Schlüssel werden hier nicht vom Sender der Nachricht erstellt sondern vom Empfänger. Wichtig ist hier, dass nur der öffentliche Schlüssel bekannt gegeben wird. Jeder Besitzer des öffentlichen Schlüssels hat nun die Möglichkeit dem Empfänger eine verschlüsselte Nachricht zukommen zu lassen. Jedoch kann nur der Empfänger solche Nachrichten auch entschlüsseln, da er die einzige Person ist, welche den privaten Schlüssel besitzt.\\\\
Das \textbf{RSA-Verfahren} ist das bekannteste und wohl am meist verbreitete Modell der Public-Key-Verschlüsselung. Dieses Verfahren wurde 1977 von Roland \textbf{R}ivest, Adi \textbf{S}hamir und Leonard \textbf{A}dleman benannt und basiert auf einem von Diffie und Hellmann entwickelten Verfahren zum sicheren Schlüsselaustausch (Rempe und Waldecker 2009: 105).

\section{Zusammenhang von Primzahlen und der Kryptographie}

\newpage

\section{Was noch fehlt und unbedingt mit eingebaut werden muss!!}
\subsection{Bézout's Lemma}
\subsection{Was ist die modulare Arithmetik?}
\subsection{Fermat's Little Theorem}
\subsection{Was ist der Algorithmus für wiederkehrendes Quadrieren?}
\subsection{Wann ist eine Nummer ein perfektes Quadrat? - Newton Methode}
\newpage

\section*{Literaturverzeichnis}
\addcontentsline{toc}{section}{\protect{}Literaturverzeichnis}
\printbibliography

\newpage

\listoftables
\addcontentsline{toc}{section}{\protect{}Tabellenverzeichnis}
\newpage

\listoffigures
\addcontentsline{toc}{section}{\protect{}Abbildungsverzeichnis}


\end{document}
